\subsection{Building of the "PingaTube" detector}
\label{sec:building_pingatube}


\subsubsection{Materials for the PingaTube}
\label{sec:materials_pingatube}

In order to build the "PingaTube" detector the following parts were used:

\begin{itemize}
  \item a cathod consisting in a aluminium tube (l = 146.5 mm, d$_{inner}$ = 20.63  mm, d$_{outer}$ = 22.9 mm, d$_{window}$ = 9.2 mm)
  \item an anode consisting in a Cu/Be wire as the anode (d = 0.025 mm)
  \item two brass tubes with the purpose of keeping uniformity on the electric field around the anode (d = 1 mm, l$_{1}$ = 40.64 mm and l$_{1}$ = 42.72 mm)
  \item insulating endcaps
  \item P10 gas Argon (90\%)/CH$_{4}$ (10\%)
  \item gas tubes in order to flux the gas (d = 2.1 mm)
  \item HV connector in order to supply high voltage to the anode
  \item a ground cable in order to connect the cathode.
  \item O-ring wire connector
  \item electric fastenning belt
  \item Ultrasound bath with 10/90 Isopropanol/Water
  \item Lab tools (Caliper (precition = 1/100 mm), Needle, Gloves, Files, Drill, Saw, Microscope, Oscilloscope, etc) and consumables (Soldering flux, aluminium tape, epoxy glue, etc)  
\end{itemize}

\subsubsection{Design and Assembly}
\label{sec:design_and_assembly_beercan}
The design was intended for using the aluminium tube as a gas chamber. Most of the materials were provided and hence creatively assembling the detector was the main task. 
Some pieces were cut by the instructors because of the lab sefety rules with the mechanical tools. Others such as the brass tubes were prepared by the students. 
Using a lab microscope it was verified that no significant notches were present on the brass tubes, which was required for the best functionability possible.
For the assembly of the detector, the parts were cleaned in an ultrasonic bath of isopropanol/water mixture.

The Cu/Be wire was soldered to the HV
connector and glued to the end caps. In order to fix and seal the chamber the end caps were glued with a two component epoxy glue.
The Cu/Be wire was inserted thru two brass tubes in both the extremes of the chamber in order to ensure as uniform field as possible. 
A window in a form of circular hole was drilled around the middle part of the chamber. The end caps have holes to provide access for gas tubes, they were glued in both sides with the epoxy glue. 
The Cu/Be wire and the pipe were connected and soldered  the high voltage connector as anode and ground respectively. A part of the aluminum tube was polished to create space for grounding and that surface was covered with copper tape for providing better contact.
 Given that the Cu/Be is toxic gloves were required for handling it. 
Additionally the wire easily bends increasing the risk of it being break. The wire was soldered to the extremes of the brass tubes already mounted in the end caps. All was sealed with the epoxy glue afterwards.


After the assembly process the P10 gas was flushed into the detector in order to extract as much O$_2$ as possible. The main reason of this is O$_2$ is a quenching gas that absorbs the avalanche electrons.
Presence of leakages of gas from the detector tube was checked with a flowmeter.

Volume of the aluminum tube:
\emph{Volume of the tube}, Volume of the tube
\begin{equation}
  \label{eq:tube_volume}
  V=\frac{\pi L^2}{4} =14.65*3.14*2.06324*2.06324=48,96 cm^3
\end{equation}

With a flow 10 ml/min = 600 cm$^3$/h content of the volume changes ~10 times per hour.

The variation of O$_{2}$ concentration with time is shown on Fig.

Units of x-axis are ppm = 1,000,000 m$_{c}$ / m$_{s}$, where 
m$_{c}$ = mass of component (argon/methane mix, kg)
m$_{s}$ = mass of solution (oxygen, kg)




Some failures which were fixed were:
\begin{itemize}
  \item one of the gas tubes did were not interted properly which caused gas leaking, The problem was sucessfuly fixed afterwards by removing and reinserting it properly. 
  This caused a lot of delay due to the time that the glue delays on drying besides the fact that it was needed to reflux the gas on the detector.
  \item There were some gas leakages that were sorted out propperly by glue sealing the positions in which the floweter detector establish the gas leakages.
  \item One of the brass tubes went loose inside of the aluminium tube. The issue was sorted out on time before the glu dried without any further problem.
\end{itemize}

\subsection{Building of the "BeerCan" detector}
\label{sec:building_beercan}

\subsubsection{Materials for the BeerCan}
\label{sec:materials_beercan}
In order to build the "BeerCan" detector the following parts were used:

\begin{itemize}
  \item a catode consisting in a beer can (l = 146.5 mm, d$_{inner}$ = 64.0 mm, d$_{thickness}$ = 0.2 mm)
  \item an anode consisting in a Cu/Be wire as the anode (d = 0.075 mm)
  \item two brass tubes. (d = 1 mm, l$_{1}$ = 40.64 mm and l$_{1}$ = 42.72 mm)
  \item guide in order to protect the brass tubes of touching the can (d = 3.21 mm)
  \item bushing
  \item plastic endcaps
  \item gas tubes in order to flux the gas (d = 2.1 mm)
  \item HV connector in order to supply high voltage to the anode 
  \item a ground cable in order to connect the cathode.
  \item P10 gas Argon (90\%)/CH$_{4}$ (10\%)  
  \item Ultrasound bath with 10/90 Isopropanol/Water
  \item Lab tools (Caliper (precition = 1/100 mm), Needle, Gloves, Files, Drill, Saw, Microscope, Oscilloscope, etc) and consumables (Soldering flux, aluminium tape, epoxy glue, etc) 
\end{itemize}
  
\subsubsection{Design and Assembly of the BeerCan}
\label{sec:design_and_assembly_beercan}
The design was intended for using a beer can as gas chamber detector. The building and assembly was very similar to the description on the PingaTube detector with the mechanical differences imposed by the beer can geometry.
Again most of the materials were provided and hence creatively assembling the detector was the main task. After the assembly process the P10 gas was flushed into the detector and the flowmeter was used for detecting gas leakages.
Some differences in the assembly of the BeerCan detector compared to the PingaTube detector were:

\begin{itemize}
  \item Necesity of scratching inner and outer surfaces of the beer can so the nonconductive cover was removed.	
  \item Use of 0.075 mm brass tubes for easier manipulation and insertion of the Cu/Be wire.
\end{itemize}
  

The brass tubes had a diameter of 5.5 mm. The length was chosen so that their end is at a
distance of 3 cm from the center of the can. Further, the diameter of the gas tubes was 2.86 mm and
the length of banana plug was 67 mm.

Volume of the beercan tube:
\emph{Volume of the tube}, Volume of the tube
\begin{equation}
  \label{eq:tube_volume}
  V=\frac{\pi L^2}{4} =14.65*3.14*6.40*6.40=471,05 cm^3
\end{equation}

With a flow 10 ml/min = 600 cm$^3$/h content of the volume changes ~10 times per hour.

The variation of O$_{2}$ concentration with time is shown on Fig.

Units of x-axis are ppm = 1,000,000 m$_{c}$ / m$_{s}$, where 
m$_{c}$ = mass of component (argon/methane mix, kg)
m$_{s}$ = mass of solution (oxygen, kg)


Some failures:
\begin{itemize}
  \item There was gas leaking in the endcaps, they were sealed by using the epoxy gum and covering all the leaking points detected. 
  \item The Cu/Be wire got blended and a little bit loose while trying to solder it to the one of the brass tubes which might have cause the detector to have an extremely poor signal hence it was considered a failure. There was not enough time for fixing this issue since it practically would have taken the dissasembly of the whole detector.
\end{itemize}    


\subsection{Measurement apparatus}
\label{sec:meas-appar}





%%% Local Variables:
%%% mode: latex
%%% TeX-master: "prop_counter"
%%% End:


Insert conclusions.
%%% Local Variables:
%%% mode: latex
%%% TeX-master: "prop_counter"
%%% End:

During the lab two proportional counters were built one using a piece of a common aluminum tube and one using a common empty beer can. 
Both detectors were assembled and filed with an Ar-CH$_4$ mixture and a high voltage was applied to their anodes. Because of assembly problems only the PingaTube detector was able to measure radiation 
from the radioactive sources. The errors on the assembly that lead to the BeerCan failure were detected but because of the time constraints it was not possible to fix them. In the case of the PingaTube detector 
the signal started being succesfully 
measured at 750 V with Gain 200 for the Fe$_{22}$. Due to time constraints it was only posible to get the 870 V with Gain 100 for the Am$_{241}$. Using Fe$_{22}$ the voltage was only increased till 1410 V with Gain 200 
because the signal coming from the spectrum amplifier was seen saturated on the osciloscope. No colimators were used during the activity because it was not needed for the purposes of the lab.
The relationship between collected charge and voltage was established and afterwards the collected charge versus MCA channel was obtained. It is estimated that the detector reaches the proportional region at around 1100 V.
The correspondent spectra of Fe$_{22}$ and Am$_{241}$ were succesfully obtained. In the case of the charge multiplication vs. Voltage​ and Resolution vs. Voltage​ plots they could only be obtained for Fe$_{22}$. 
The resolution of the detector was considerably good, below 20\% and oscilating in around 16\% for voltages above 1050 V.